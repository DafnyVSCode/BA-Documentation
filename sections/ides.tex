\subsection{Support for other IDEs} \label{ides}
Since the plugin is structured as a language server, it should theoretically be possible to integrate it into any IDE which implements the language server protocol without any problems. In reality, some customization is often needed and some popular IDEs also dont fully implemenet the protocol yet. This chapter details which IDEs were looked at as possible hosts for the plugin and what would have to be done for a complete integration.

\subsubsection{Eclipse integration}
The eclipse project \href{https://projects.eclipse.org/projects/technology.lsp4e}{LSP4E} aims to integrate existing language servers into the Eclipse IDE in a easy way. 
\newline
"It includes some APIs to turn language server protocol elements into Eclipse IDE concepts and a generic integration allowing to easily plug any language server to an Eclipse IDE instance without need to write Java code, either via a plugin associating a new language server, and by letting users manually bind language servers to their IDE." \cite{lsp4e}

It is built on top of \href{https://github.com/eclipse/lsp4j}{LSP4J}, a Java implementation of the language server protocol. 
\newline
Integrating the Visual Studio Code Dafny language server into Eclipse could become possible in the future. Right know there is no way to interact on the client side as well. One only can specify a language server based on a program (NodeJS) and arguments (extension.js) which is executed and used for querying information. One can't add any behavior to Eclipse itself, which would be necessary for certain features. Also, the sendRequest isn't implemented yet (Commit: 1615e07), which is important for starting the DafnyServer correctly. 
The better way would be to program a new Eclipse plugin, based on LSP4J, which then would also allow to customize the status bar, run scripts and show progress information. 


\subsubsection{EMACS integration}