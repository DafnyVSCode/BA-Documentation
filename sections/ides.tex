\subsection{Support for other IDEs} \label{ides}
Since the plugin is structured as a language server, it should theoretically be possible to integrate it into any IDE which implements the language server protocol without any problems. In reality, some customization is often needed and some popular IDEs also do not fully implement the protocol yet. This chapter details which IDEs were looked at as possible hosts for the plugin and what would have to be done for a complete integration.
\subsubsection{Eclipse integration}
The eclipse project \href{https://projects.eclipse.org/projects/technology.lsp4e}{LSP4E} aims to integrate existing language servers into the Eclipse IDE in an easy way. 
\newline
"It includes some APIs to turn language server protocol elements into Eclipse IDE concepts and a generic integration allowing to easily plug any language server to an Eclipse IDE instance without need to write Java code, either via a plugin associating a new language server, or by letting users manually bind language servers to their IDE." \cite{lsp4e}
It is built on top of \href{https://github.com/eclipse/lsp4j}{LSP4J}, a Java implementation of the language server protocol. 
\newline
Integrating the Visual Studio Code Dafny language server into Eclipse could become possible in the future. Right know there is no way to interact on the client side. One only can specify a language server based on a program (NodeJS) and arguments (extension.js) which is executed and used for querying information. One cannot add any behavior to Eclipse itself, which would be necessary for certain features. Also, the sendRequest protocol specification is not implemented yet (Commit: 1615e07), which is important for starting the DafnyServer correctly. 
The better way would be to program a new Eclipse plugin, based on LSP4J, which then would also allow to customize the status bar, run scripts and show progress information. 
\subsubsection{Emacs integration}
Emacs \cite{GNU} is a versatile IDE which enjoys great popularity especially in the open source community. Written in LISP, Emacs traditionally supported language integration via so called modes, which are demons that run in the background. This is already a similar architecture as used in a language server integration. \newline
Work on integrating language servers into Emacs has already be done. The project emacs-lsp \cite{emacsLsp} aims to provide the connection between Emacs and language server. The project itself is structured as a classical Emacs demon and allows interactions with a language server. There already exist some integrable language servers in languages such as Java or Haskell. The integration of emacs-lsp and existing language servers seems to be pretty trivial, an example can be seen in \cite{javaEmacs} using Java. \newline
Since the language server allows custom messages to be defined as detailed in the implementation documentation, the Dafny plugin defines some of them. One set of custom messages is necessary, since the plugin allows for downloading the Dafny compiler automatically. The communication in regard to this feature must be extended to the IDE, so that the user is aware of this feature. Additional custom components are the state of the file so that the programmer sees if it is verified or not, or the displacement of counter examples for proofs.\newline
These features were deemed very useful in this project, such that they should be a part of every IDE integration. This means that a little wrapper would have to be written on the client side of the language server, which understands the custom messages and can relay this information to the IDE itself. The work that has to be done is trivial, since it simply means to pass information on and display them accordingly. \newline
It was decided that gaining a working knowledge of LISP and Emacs in order to write such a wrapper was not in the scope of this project. For a programmer familiar with Emacs and LISP this task should take no more than a week. The custom messages that need to be implemented can be found in the implementation documentation. It can be argued that this extension should be one of the first ones to be tackled by later work, as one can gain a considerably larger user base without having to invest too much work. 
\subsubsection{Monaco integration}
Monaco\cite{monaco} is the code editor that powers Visual Studio Code. It is also possible to use Monaco as a standalone editor in the browser. It would be intriguing to integrate Dafny directly into Monaco, since there are not many simpler setups imaginable as opening a browser window. It was therefore decided to look into a possible integration of the plugin during the course of this project. \newline
It soon became apparent that the project does not seem very active, as the latest commit was two months old at the time of this writing. The documentation for developers is also very sparse. There was hope that an integration would be trivial, since the plugin is structured as a language server. However, Monaco does not implement the language server protocol. In the documentation, it can even be read that "Extensions written for Visual Studio Code will not run in Monaco"\cite{monaco}, without any further explanation given. \newline
However, integrating a new language into Monaco is possible, as can be seen in an example for Typescript in  \cite{monacoType}. However, the setup is different than a language server integration or a Visual Studio Code plugin. It is questionable how much code could be shared. While the idea of running Dafny in an editor in the browser is interesting, the work that would have to be done in order to achieve this was deemed to be outside of the scope of this project. In addition, if further work aims to implement this integration, an evaluation on how active the work on Monaco is should be done first, given the apparent standstill in the project at the time of this writing.