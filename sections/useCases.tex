\section{Use Cases}
\subsection{Use Case Diagram}
\begin{figure}[h]
	\centering
	\includegraphics[width=1\linewidth, height=5in]{"./img/UseCase"}
	\caption{Use Case Diagram}
	\label{fig:usecase-diagram}
\end{figure}
\subsection{Actors and Stakeholder}
\begin{itemize}
	\item Programmers
	\item Microsoft
\end{itemize}
\subsection{Descriptions (brief)}
\subsubsection{UC1: Easy installation of Dafny plugin}
A programmer can simply install the Dafny plugin by running an automatic installer which sets all path variables and makes additional needed environment adjustments. This can be done on Windows 10 in a .NET environment or either on Linux or OSX in a mono environment.
\subsubsection{UC2: Syntax Highlighting}
The system automatically does syntax highlighting while the user writes a Dafny (dfy.) file.
\subsubsection{UC3: Reporting of Dafny best practices violations}
The plugin can be configured with a simple DSL config file which describes common best practices for Dafny. The plugin reports violations of these rules via the standard Visual Studio Code notification mechanisms.
\subsubsection{UC4: Automatic generation of contracts}
The plugin can be configured with a simple DSL-File to recognize certain situations, which could benefit from the setting of Pre- and Postconditions. The plugin offers to add these in form of a refactoring via the common Visual Studio Code mechanisms.
\todonl{Maybe remove dsl from description and add annotation }
\subsubsection{UC5: Autocompletion for identifiers}
The plugin offers autocompletion of Dafny code while the user types via the standard Visual Studio Code mechanisms.
\subsection{Descriptions (fully dressed)}

\subsubsection{UC1: Easy installation of Dafny plugin}
\rowcolors{1}{gray!25}{white}
\begin{longtable}{l | p{0.7\textwidth} }
	Description & A programmer can simply install the Dafny plugin by running an automatic installer which sets all path variables and makes additional needed environment adjustments.\\ \hline
	Primary Actor & Programmer\\ \hline
	Trigger & Programmer wants install the Dafny plugin to visual studio code.\\ \hline
	Stakeholder and Interests & Programmer: Wants an easy, automated installation of the plugin. \newline Microsoft: Wants a stable Dafny integration to fulfill the needs of its clients.\\ \hline
	Preconditions &\begin{itemize}
		\item Depending on the environment, either the .NET or the mono framework are installed.
		\item Visual Studio Code is installed.
		\item The programmer has admin privileges in his environment.
	\end{itemize}\\ \hline
	Postconditions & 
	\begin{itemize}
		\item The plugin works without problems, the programmer can start writing code.
	\end{itemize} \\ \hline
	Main Success Scenario & 
	1. Programmer downloads the plugin via the Visual Studio Code plugin store. \newline
	2. The plugin determines which platform it is run on, and sets either the path to the .Net framework or the mono framework. \newline
	3. The plugin downloads Dafny and sets the path to the Dafny server binary. \newline
	4. The plugin then installs itself via the standard Visual Studion Code plugin mechanisms. \newline
	5. The plugin prompts a success message and the programmer is ready to code in Dafny.\\ \hline
	Extensions & 
	2a. The installer can't find, depending on the environment, either the mono or the .Net framework in the standard locations. \newline 
	- It prompts the user to enter the location and does so until the framework is found. \\ \hline
	Special Requirements & None\\ \hline
	Frequency of Occurrence & Usually once per working environment\\ \hline
	Open Issues & None \\ \hline
	\caption{UC1}
\end{longtable}

\subsubsection{UC2: Syntax Highlighting}
\begin{longtable}{l | p{0.7\textwidth} }
	Description & The system automatically does syntax highlighting while the user writes a Dafny (dfy.) file.\\ \hline
	Primary Actor & Programmer\\ \hline
	Trigger & Programmer writes Dafny code in Visual Studio Code.\\ \hline
	Stakeholder and Interests & Programmer: Wants enhanced readability for the source files he is working on. \newline Microsoft: Wants a state of the art IDE integration of Dafny.\\ \hline
	Preconditions &
	\begin{itemize}
		\item Visual Studio Code with the Dafny plugin installed is running.
		\item Dafny code is being written in a .dfy file.
	\end{itemize}\\ \hline
	Postconditions &
	\begin{itemize}
		\item The source code is highlighted in different colors according to common standards in general and the Visual Studio Code guidelines specifically.
	\end{itemize}\\ \hline
	Main Success Scenario & 
	1. Programmer types code into a .dfy file. \newline
	2. The plugin detects the changes and applies syntax highlighting through the standard Visual Studio Code mechanisms.\\ \hline
	Extensions & 
	2a. The newly written code causes a compilation error and cannot be interpreted. \newline 
	- The previous syntax highlighting stays in place, the errors are highlighted according to common practices with compilation errors in Visual Studio Code. \\ \hline
	Special Requirements & None\\ \hline
	Frequency of Occurrence & Very often, after every key up event.\\ \hline
	Open Issues & None \\ \hline
	\caption{UC2}
\end{longtable}

\subsubsection{UC3: Reporting of Dafny best practices violations}
\begin{longtable}{l | p{0.7\textwidth} }
	Description & The plugin can be configured with a simple DSL config file which describes common best practices for Dafny. The Plugin reports violations of these rules via the standard Visual Studio Code notification mechanisms.\\ \hline
	Primary Actor & Programmer\\ \hline
	Trigger & Programmer writes Dafny code in Visual Studio Code.\\ \hline
	Stakeholder and Interests & Programmer: Wants to write the cleanest code possible using common Dafny idioms. \newline Microsoft: Wants to support programmers getting the most out of Dafny.\\ \hline
	Preconditions &
	\begin{itemize}
		\item Visual Studio Code with the Dafny plugin installed is running.
		\item Dafny code is being written in a .dfy file.
	\end{itemize}\\ \hline
	Postconditions &
	\begin{itemize}
		\item Violations of common best practices for Dafny are reported through the standard Visual Studio Code mechanisms.
	\end{itemize}\\ \hline
	Main Success Scenario & 
	1. The plugin is installed preconfigured with a collection of common best practices for Dafny, which is done through a DSL file. \newline
	2. Programmer types code into a .dfy file. \newline 
	3. The plugin continuously checks for violations of the rules. \newline 
	4. If a violation is detected, it is reported through the standard mechanisms of Visual Studio Code.\\ \hline
	Extensions & 
	1a. The predefined rules are not sufficient for the programmer. \newline 
	- The programmer can updated the configuration file himself to include his own or his company's coding guidelines. \\ \hline
	Special Requirements & None\\ \hline
	Frequency of Occurrence & Very often, after new valid syntax was written.\\ \hline
	Open Issues & None \\ \hline
	\caption{UC3}
\end{longtable}

\subsubsection{UC4: Automatic generation of contracts}
\begin{longtable}{l | p{0.7\textwidth} }
	Description & The plugin can be configured with a simple DSL-File to recognize certain situations which could benefit from the setting of Pre- and Postconditions. The plugin offers to add these in form of a refactoring via the common Visual Studio Code mechanisms.\\ \hline
	Primary Actor & Programmer\\ \hline
	Trigger & Programmer writes Dafny code in Visual Studio Code.\\ \hline
	Stakeholder and Interests & Programmer: Wants help to find appropriate contracts. \newline Microsoft: Wants to support programmers getting the most out of Dafny.\\ \hline
	Preconditions &
	\begin{itemize}
		\item Visual Studio Code with the Dafny plugin installed is running.
		\item Dafny code is being written in a .dfy file.
	\end{itemize}\\ \hline
	Postconditions &
	\begin{itemize}
		\item Specification constructs suitable for the context were added to the code.
	\end{itemize}\\ \hline
	Main Success Scenario & 
	1. The plugin is installed preconfigured with a collection of common situations for Dafny and their corresponding specification constraints, which is done through a DSL file. \newline
	2. Programmer types code into a .dfy file. \newline 
	3. The plugin continuously checks for situations that could benefit form the setting of specification constraints. \newline 
	4. If such a semantic is detected, it is reported through the standard mechanisms of Visual Studio Code, with a command offered to add the constraints. \newline
	5. The programmer invokes the refactoring and the necessary code is added.\\ \hline
	Extensions & 
	1a. The predefined situations and contract code additions are not sufficient for the programmer. \newline 
	- The programmer can updated the configuration file himself to include support for his own or his company's  common code semantics. \\ \hline
	Special Requirements & None\\ \hline
	Frequency of Occurrence & Very often, upon typing code with no syntax errors.\\ \hline
	Open Issues & None \\ \hline
	\caption{UC4}
\end{longtable}

\subsubsection{UC5: Autocompletion for identifiers}
\begin{longtable}{l | p{0.7\textwidth} }
	Description & The plugin offers autocompletion of Dafny code while the user types via the standard Visual Studio Code mechanisms.\\ \hline
	Primary Actor & Programmer\\ \hline
	Trigger & Programmer writes Dafny code in Visual Studio Code.\\ \hline
	Stakeholder and Interests & Programmer: Wants to be more productive while writing Dafny code. \newline Microsoft: Wants a state of the art IDE integration of Dafny.\\ \hline
	Preconditions &
	\begin{itemize}
		\item Visual Studio Code with the Dafny plugin installed is running.
		\item Dafny code is being written in a .dfy file.
	\end{itemize}\\ \hline
	Postconditions &
	\begin{itemize}
		\item An identifier was autocompleted.
	\end{itemize}\\ \hline
	Main Success Scenario & 
	1. Programmer types code into a .dfy file. \newline
	2. The plugin detects the changes and searches for the beginning of known identifiers. \newline
	3. If such a beginning is found, completion if offered through the standard mechanisms of Visual Studio Code.\\ \hline
	Extensions & 
	None. \\ \hline
	Special Requirements & None\\ \hline
	Frequency of Occurrence & Very often, after every key up event.\\ \hline
	Open Issues & None \\ \hline
	\caption{UC5}
\end{longtable}