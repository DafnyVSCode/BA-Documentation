\subsection{Goals Reached}\label{golrech}
This chapter details the goals that were reached in this project and puts them into contrast with the current solutions that were introduced in \ref{cursol}.
\subsubsection{Platform Independence}
The chapter \ref{setup} details how the plugin runs on all major platforms. While only one existing solution, namely the one offered for Emacs, runs on all platforms, the user still has to make sure to have the Dafny pipeline configured correctly for his environment. \newline
This project has developed the only solution which is able to run the whole pipeline on all operating systems and does so automatically. This contributes in reaching the biggest user base possible and invites all programmers to try out Dafny without having to switch the environment that they have become accosted to.  
\subsubsection{Setup}
In this area, the product developed during the course of this project surpasses all existing solutions. While those all require the user to install and configure the Dafny platform, this plugin handles the whole setup automatically. The way this is done is detailed in \ref{setup}. The Visual Studio Code plugin can be installed by one click within the marketplace that Microsoft offers, while the whole dependency resolution resides in the language server part of the plugin, meaning this automatic installation is also possible when integrating with other IDEs. \newline
Having unit tests in place which hinder breaking changes and doing all the resolution in a central place lays the foundation of a setup that is as simple as can be. This invites programmers to try out Dafny without having to invest any energy and is an important part in widening the user base. 
\subsubsection{Usability}
The language server protocol details both standard and custom messages than can be emitted by the server. This was used during the work on this project to relay almost all information emitted from Dafny to the programmer. The plugin also translated this messages in idiomatic Visual Studio Code notifications, giving them the appropriate form or displaying them in the correct GUI element. It can therefor be said that this project leverages all elements that Visual Studio Code offers and programmers familiar with the most common GUI driven IDEs should have no problem working all the features that were implemented. \newline
While already all existing solutions integrate nicely into their respective IDEs, only the Visual Studio integration details the same amount of information as the plugin developed in this project. This makes it attractive for a wide range of programmers that are used to working with different IDEs. 
\subsubsection{IDE Independence}
While developing this plugin, it was decided to implement it as a language server. This has the potential to decouple the plugin from a concrete IDE and make it reusable, as was detailed in \ref{setup}. Making such an integration to an IDE other than Visual Studio Code work was outside of the scope of this project, but a proof of concept was done for Emacs. \newline
The proof of concept and the fact that the protocol is gaining traction fast make it safe to say that this project hat laid the foundation for many potential IDE integrations to come. \ref{ides} gives some ideas which integrations would be feasible in the near future. \newline
The approach chosen when designing this plugin as a language server is unique among the existing solutions, making it the only one that is not hardwired to a specific IDE. This has great potential of widening the Dafny audience, but also of inviting other programmer to contribute to the plugin.
\subsubsection{Feature Richness}
During this project, many different feature were implemented. This includes language agnostic features detailed in \ref{agfeatures} as well as features that lend Dafny specific support to the programmer detailed in \ref{dffeatures}. While it was an important concern that programmers can use most features that they are used to from working with other languages, much thought was spent on how programmers can be supported writing Dafny specific code. This resulted in a collection of some solutions that are unique to this project. \newline
Comparing the stand of the project with existing solutions, it surpasses most vastly regarding feature richness in both areas. The only platform to offer features that this project does not do, is the Visual Studio integration. These features include a Dafny debugger, which is very helpful but sadly was outside of the scope of this project. However, this project integrates some features that have not been implemented yet in any IDE, especially automatic specification construct generation for some situations. \newline
Since most solutions are open source, it stands to hope that they continuously offer more features by getting inspiration from other solutions. At the current time, programmers that make use of the solution developed in this project benefit from one of the two most feature rich Dafny programming experiences. 



