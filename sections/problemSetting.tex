\section{Outline}
\subsection{The problem and its setting}
This chapter presents the background of the study, problem and its significance, and the scope and deliminition of the study.
\subsubsection{Introduction}
Dafny is a language designed and implemented by Microsoft Research. It offers built-in specification constructs. These include pre- and postconditions, frame specifications as well as termination metrics. Further support such as ghost variables and recursive functions are also implemented. Through such specification primitives, the Dafny verifier, invoked during compilation, can be used to verifiy the specified aspects of the functional correctness of the program. \newline
Dafny is typically used via its Visual Studio IDE integration under the Windows operating system. Dafny can be run from the command line and a Visual Studio integration exists. This integration verifier therefor allows for an efficient workflow of editing a program while constantly being given feedback about its functional correctness. The Dafny compiler and verifier can additionally be invoked from the command line. \newline
Microsoft would like to integrate of Dafny into the cross-platform Visual Studio Code IDE. Work on this has already been started through a plugin by Jonathan  Rionatan. It currently works within the mono-environment and provides feedback from the verifier. \newline

\subsubsection{Statement of the problem}
This thesis aimes to research on how Dafny programmers can be best supported during their work and incorporate these findings in a production quality plugin for Visual Studio Code. 
\subsubsection{Significance of study}
Since the goal of the thesis is a production quality plugin, the thesis aims to immidiately simplify the way programmers use Dafny. Through Dafny's strong functional correctness facilities and Visual Studio Code's simple and modular architecture there is a chance that the plugin could guide developers through the transition to world of multi-core and multi-threaded applications, which are becoming more and more popular. Also in high risk domains the use of a specification constructs is of great benefit for the developers, which the developed plugin aims to make easier to harvest.
\subsubsection{Scope and delimination}
The plugin is limited to be used in three defined environments, although they compromise a huge percentage of environments used in programming. The plugin offers a fixed set of features which are detailed in this thesis, but remains open for adaption and extension.
\subsubsection{Definition of terms}
