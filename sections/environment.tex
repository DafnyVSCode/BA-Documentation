\subsection{Environment}\label{environment}
This chapter details the underlying structure of the plugin which provides a basis to implement the individual concrete features.

\subsubsection{Symbol Service}\label{symbolservice}
The different features often need detailed information about symbols and their relations. A symbol table is a data structure used by a compiler to keep track of scope / binding information about names. These names are used in the source program to identify the various program elements, like variables, constants, procedures, and the labels of statements.\cite[239]{compiler} \newline
With regard to performance, but also to reduce overhead it was decided to implement a symbol service in the language server part of the plugin. It's main goal is to cache a subset of the symbol table of the compiler, such that features which need information about names in the code can use the symbol service to gain this insight without having to invoke the compiler every time a lookup has to be performed or each feature having to handle the information caching separately \newline
Since compilation is a heavy task, results should be cached efficiently. There is a trade off here though with the validity of the symbols, since if they are cashed too long, they don't represent the code anymore. The solution chosen is a lazy loading approach, meaning whenever a component queries the service the first time, the symbols are loaded and cached for the first time. This is usually done through by codeLenses, since they need the symbol information and they are created every time a file is opened right at the start. To make sure that no unnecessary loads are performed, the symbol table for each document is stored alongside a hash of the content of the document. Before sending a new request to DafnyServer, a comparison of the hashes is done to ascertain that the file really has changed. In future, this would also allow efficient caching by using histories of symbol tables when actions are undone. Next to the symbol table, it also stores the supplied text document itself, allowing for quick text manipulation if the content has not changed when loading the symbols.\newline
The next consideration was on how to store the symbol table and which information should be saved. The fields used are mostly needed to either determine which kind of symbol it is, where its scope is, its fully qualified name and relationships in form of references are also stored. This lead to the following data structure that is stored in the symbol service, the example is simplified for better readability: \newline
\newline\newline
\textbf{Symbol Tables: }
\begin{lstlisting}[language=json,firstnumber=1]
[
 {
  fileName: "filepathInUriFormat",
  hash: -483616355,
  symbols: [
   {
    call: null,
    column: 5,
    document: "filepathInUriFormat",
    end: Range(8, 12),
    ensures: [...],
    line: 8,
    module: "module",
    name: "balance",
    parentClass: "BankAccountUnsafe",
    position: 96,
    range(start, end),
    References: [
     {
      column: 4,
      document: "filepathInUriFormat",
      end: Range(11, 11),
      line: 11,
      methodName: "balance",
      position: 154,
      range: Range(start, end),
      referencedName: "balance"
     },
     ...
    ],
    requires: [...],
    start: Range(8, 5),
    symbolType: "Field"
   },
   ...
  ]
 },
 ...
]
\end{lstlisting}

The possible symbol types that are stored are defined as follows:
\begin{lstlisting}[language=json,firstnumber=1]
{
 Unknown,
 Class,
 Method,
 Function,
 Field,
 Call,
 Declaration	
}

\end{lstlisting}
The symbol table offers the following API to obtain symbols: \newline
\paragraph{addSymbols(doc: Textdocument, symbols: SymbolTable, forceAddition: boolean=false): void} This saves the supplied symbol table and associates it with the text document given. The default behavior is to get a new symbol table from the DafnyServer anyway and compare if they have changed. If so, the never one is chosen and persisted. If the parameter forceAddition is set to true, the symbol table is stored even though it might be out of date, a new version is not queried.

\paragraph{getTextDocument(uri: string): TextDocument} If the service has cached a text document specified by the Uri supplied, it returns it.

\paragraph{getSymbols(doc: TextDocument): Promise<SymbolTable[]>} Returns all the symbol tables that are stored in the symbol service. Optionally, a text document can be given as an argument. If the symbols to this document are not cached, then they are queried from the DafnyServer. Since this is an asynchronous operation, the return type is a promise of symbol tables. This method is useful when actions have to be done across the whole workspace.

\paragraph{getAllSymbols(doc: TextDocument): Promise<Symbol[]>} Similar to the method above, but the result is already flattened to an array of symbols across the whole code base. This is useful when it is not important to work with the symbols on a file per file basis.

\paragraph{getSingleSymbols(doc: TextDocument): Promise<SymbolTable>} This method allows for gaining the symbols for a single text document. If they are not cached yet, the service queries the Dafny Server for it, stores the result and then returns it. Since this operation is asynchronous, the return type is a promise.
\newline
When querying the the DafnyServer, the server defensively starts the communication with it by supplying the symbol verb, the file path and content for which it wants symbols on stdout. It then waits on the socket for a response, and if it is well formed and contains the queried symbols, the JSON returned by the server is parsed and the new symbol elements are constructed out of the JSON, which are then saved in the service. \newline
When there is an error, for instance a connection error, or there was a compilation error which prevents the building of a symbol table, the service deals with the error and just keeps, if it has any, the old symbol table of the file, so that the data is always in the most consistent state possible. Since compilation errors or connection errors are signaled to the user, the service listens until the broken elements have been repaired and then queries the server again for the symbols. \newline
Parsing of a successful response is also done conservatively, meaning that if an important property on for instance a method symbol is missing, this symbol is not stored, although all other valid symbols from that batch are stored. This allows for the maximum of analysis with only partly correct data. \newline
An improvement to the symbol service could be to do the caching more cleverly, for instance ignoring white space changes. A trade off in this area is that the calculations to decide if an update should be made could be more expensive then the update itself, so this should be monitored closely. Also when more and more complex refactoring and analysis should be done with the plugin, the data structure stored must probably be expanded. The extreme would be to save the whole AST in the symbol service which off course would be an overkill. Also here a balance thusly must be found between information richness and scope. Another consideration could be to optimize the lazy loading approach, for instance draw on a simple heuristic which files might be opened soon and load the symbols for them preemptively\newline
