\section{Use Cases}
\subsection{Use Case Diagramm}
\subsection{Aktoren und Stakeholder}
\begin{itemize}
	\item Programmers
	\item Microsoft
\end{itemize}
\subsection{Descriptions (brief)}
\subsubsection{UC1: Windows Installation}
A programmer can install the Dafny Plugin on his .Net environment running on Windows 10.
\subsubsection{UC2: Linux Installation}
A programmer can install the Dafny Plugin on his mono environment running on a common Linux distribution such as Ubuntu 16.04.
\subsubsection{UC3: OSX Installation}
A programmer can install the Dafny Plugin on his mono environment running on OSX.
\subsubsection{UC4: Easy installation of Dafny plugin}
A programmer can simply install the Dafny plugin by running an automatic installer which sets all path variables and makes additional needed environment adjustments.
\subsubsection{UC5: Syntax Highlighting}
The system automatically does syntax highlighting while the user writes a Dafny (dfy.) file.
\subsubsection{UC6: Compilation of Dafny best practices and reporting of these rules}
The plugin can be configured with a simple json-like config file which describes common best practices for Dafny. The Plugin reports violations of these rules via the standard Visual Studio Code notification mechanisms.
\subsubsection{UC7: Automatic generation of contracts}
The plugin can be configured with a simple DSL-File to recognize certain semantics, which could benefit from the setting of Pre- and Postconditions. The plugin offers to add these in form of a refactoring via the common Visual Studio Code mechanisms.
\subsubsection{UC8: Autocompletion for identifiers}
The plugin offers autocompletion of Dafny code while the user types via the standard Visual Studio Code mechanisms.
\subsection{Descriptions (fully dressed)}
\subsubsection{UC1: Windows Installation}
\rowcolors{1}{gray!25}{white}
\begin{longtable}{l | p{0.7\textwidth} }
	Description & A programmer can install the Dafny Plugin on his .Net environment running on Windows 10.\\ \hline
	Primary Actor & Programmer\\ \hline
	Trigger & Programmer wants to programm Dafny in Visual Studio Code.\\ \hline
	Stakeholder and Interests & 
	\begin{itemize}
		\item Programmer: Wants stable support from the IDE while developing Code.
		\item Microsoft: Wants a stable Dafny integration to fulfill the needs of its clients.
	\end{itemize} \\ \hline
	Preconditions & 
 \begin{itemize}
		\item Windows 10 Environment with a modern version of the .Net Framework installed.
	\end{itemize} \\ \hline
	Postconditions & 
	\begin{itemize}
		\item The plugin works without problems, the programmer can start writing code.
	\end{itemize} \\ \hline
	Main Success Scenario & 1. Programmer downloads the plugin. \newline
	2. During an automated installation routine, the plugin installs itself and all needed environmental changes are made. (UC4) \newline
	3. Visual Studio code is updated to use the plugin, a success message is shown\\ \hline
	Extensions & 
	2a. The installer notices that some dependencies are missing. \newline 
	- It notifies the programmer accordingly.\\ \hline
	Special Requirements & None\\ \hline
	Frequency of Occurence & Usually once per working environment\\ \hline
	Open Issues & None \\ \hline
	\caption{UC1}
\end{longtable}
\subsubsection{UC2: Linux Installation}
\begin{longtable}{l | p{0.7\textwidth} }
	Description & A programmer can install the Dafny Plugin on his mono environment running on a common Linux distribution such as Ubuntu 16.04.\\ \hline
	Primary Actor & Programmer\\ \hline
	Trigger & Programmer wants to programm Dafny in Visual Studio Code.\\ \hline
	Stakeholder and Interests & 
	\begin{itemize}
		\item Programmer: Wants stable support from the IDE while developing Code.
		\item Microsoft: Wants a stable Dafny integration to fulfill the needs of its clients.
	\end{itemize} \\ \hline
	Preconditions & 
	\begin{itemize}
		\item A modern common Linux distribution such Ubuntu 16.04 or Fedora with the mono framework installed.
	\end{itemize} \\ \hline
	Postconditions &
	\begin{itemize}
		\item The plugin works without problems, the programmer can start writing code.
	\end{itemize} \\ \hline
	Main Success Scenario & 1. Programmer downloads the plugin. \newline
	2. During an automated installation routine, the plugin installs itself and all needed environmental changes are made. (UC4) \newline
	3. Visual Studio code is updated to use the plugin, a success message is shown\\ \hline
	Extensions &
	2a. The installer notices that some dependencies are missing. \newline
	- It notifies the programmer accordingly.\\ \hline
	Special Requirements & None\\ \hline
	Frequency of Occurence & Usually once per working environment\\ \hline
	Open Issues & None \\ \hline
	\caption{UC2}
\end{longtable}

\subsubsection{UC3: OSX Installation}
\begin{longtable}{l | p{0.7\textwidth} }
	Description & A programmer can install the Dafny Plugin on his mono environment running on OSX.\\ \hline
	Primary Actor & Programmer\\ \hline
	Trigger & Programmer wants to programm Dafny in Visual Studio Code.\\ \hline
	Stakeholder and Interests & 
	\begin{itemize}
		\item Programmer: Wants stable support from the IDE while developing Code.
		\item Microsoft: Wants a stable Dafny integration to fulfill the needs of its clients.
	\end{itemize} \\ \hline
	Preconditions &
	\begin{itemize}
		\item A modern version of the OSX OS installed such as canberra with the mono framework installed.
	\end{itemize} \\ \hline
	Postconditions &
	\begin{itemize}
		\item The plugin works without problems, the programmer can start writing code.
	\end{itemize} \\ \hline
	Main Success Scenario & 1. Programmer downloads the plugin. \newline
	2. During an automated installation routine, the plugin installs itself and all needed environmental changes are made. (UC4) \newline
	3. Visual Studio code is updated to use the plugin, a success message is shown\\ \hline
	Extensions & 
	2a. The installer notices that some dependencies are missing. \newline 
	- It notifies the programmer accordingly.\\ \hline
	Special Requirements & None\\ \hline
	Frequency of Occurence & Usually once per working environment\\ \hline
	Open Issues & None \\ \hline
	\caption{UC3}
\end{longtable}

\subsubsection{UC4: Easy installation of Dafny plugin}
\begin{longtable}{l | p{0.7\textwidth} }
	Description & A programmer can simply install the Dafny plugin by running an automatic installer which sets all path variables and makes additional needed environment adjustments.\\ \hline
	Primary Actor & Programmer\\ \hline
	Trigger & Programmer wants install the Dafny plugin to visual studio code.\\ \hline
	Stakeholder and Interests & Programmer: Wants an easy, automated installation of the plugin. \newline Microsoft: Wants a stable Dafny integration to fulfill the needs of its clients.\\ \hline
	Preconditions &\begin{itemize}
		\item Depending on the environment, the preconditions of UC1, UC2 or UC3 are satisfied.
		\item Visual Studio Code is installed.
		\item The Dafny server binary and it's dependencies are installed.
		\item The programmer has admin priviledges in his environment.
	\end{itemize}\\ \hline
	Postconditions & The plugin works without problems, the programmer can start writing code.\\ \hline
	Main Success Scenario & 
	1. Programmer downloads the plugin via the Visual Studio Code plugin store. \newline
	2. The plugin determines which platform it is run on, and sets either the path to the .Net framework or the mono framework. \newline
	3. The plugin sets the path to the Dafny server binary. \newline
	4. The plugin then installs itself via the standard Visual Studion Code plugin mechanisms. \newline
	5. The plugin prompts a success message and the programmer is ready to code in Dafny.\\ \hline
	Extensions & 
	2a. The installer can't, depending on the enviroment, either the mono or the .Net framework in the standard locations. \newline 
	- It prompts the user to enter the location and does so until the framework is found. \newline
	3a. The installer can't find the Dafny server binaries in the path. \newline
	- It prompts the user to enter the location and does so until the framework is found.\\ \hline
	Special Requirements & None\\ \hline
	Frequency of Occurence & Usually once per working environment\\ \hline
	Open Issues & None \\ \hline
	\caption{UC4}
\end{longtable}

\subsubsection{UC5: Syntax Highlighting}
\begin{longtable}{l | p{0.7\textwidth} }
	Description & The system automatically does syntax highlighting while the user writes a Dafny (dfy.) file.\\ \hline
	Primary Actor & Programmer\\ \hline
	Trigger & Programmer wants install the Dafny plugin to visual studio code.\\ \hline
	Stakeholder and Interests & Programmer: Wants an easy, automated installation of the plugin. \newline Microsoft: Wants a stable Dafny integration to fulfill the needs of its clients.\\ \hline
	Preconditions &\begin{itemize}
		\item Depending on the environment, the preconditions of UC1, UC2 or UC3 are satisfied.
		\item Visual Studio Code is installed.
		\item The Dafny server binary and it's dependencies are installed.
		\item The programmer has admin priviledges in his environment.
	\end{itemize}\\ \hline
	Postconditions & The plugin works without problems, the programmer can start writing code.\\ \hline
	Main Success Scenario & 
	1. Programmer downloads the plugin via the Visual Studio Code plugin store. \newline
	2. The plugin determines which platform it is run on, and sets either the path to the .Net framework or the mono framework. \newline
	3. The plugin sets the path to the Dafny server binary. \newline
	4. The plugin then installs itself via the standard Visual Studion Code plugin mechanisms. \newline
	5. The plugin prompts a success message and the programmer is ready to code in Dafny.\\ \hline
	Extensions & 
	2a. The installer can't, depending on the enviroment, either the mono or the .Net framework in the standard locations. \newline 
	- It prompts the user to enter the location and does so until the framework is found. \newline
	3a. The installer can't find the Dafny server binaries in the path. \newline
	- It prompts the user to enter the location and does so until the framework is found.\\ \hline
	Special Requirements & None\\ \hline
	Frequency of Occurence & Usually once per working environment\\ \hline
	Open Issues & None \\ \hline
	\caption{UC5}
\end{longtable}